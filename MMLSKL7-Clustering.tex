Clustering with K-Means
In the previous chapters we discussed supervised learning tasks; we examined
algorithms for regression and classification that learned from labeled training data. In
this chapter we will discuss an unsupervised learning task called clustering. Clustering
is used to find groups of similar observations within a set of unlabeled data. We will
discuss the K-Means clustering algorithm, apply it to an image compression problem,
and learn to measure its performance. Finally, we will work through a semi-supervised
learning problem that combines clustering with classification.
Recall from Chapter 1, The Fundamentals of Machine Learning, that the goal of
unsupervised learning is to discover hidden structure or patterns in unlabeled training
data. Clustering, or cluster analysis, is the task of grouping observations such that
members of the same group, or cluster, are more similar to each other by a given
metric than they are to the members of the other clusters. As with supervised learning,
we will represent an observation as an n-dimensional vector. For example, assume that
your training data consists of the samples plotted in the following figure:
www.it-ebooks.info
Clustering with K-Means
[ 116 ]
Clustering might reveal the following two groups, indicated by squares and circles:
Clustering could also reveal the following four groups:
www.it-ebooks.info
Chapter 6
[ 117 ]
Clustering is commonly used to explore a dataset. Social networks can be clustered
to identify communities and to suggest missing connections between people. In
biology, clustering is used to find groups of genes with similar expression patterns.
Recommendation systems sometimes employ clustering to identify products or
media that might appeal to a user. In marketing, clustering is used to find segments
of similar consumers. In the following sections, we will work through an example of
using the K-Means algorithm to cluster a dataset.
Clustering with the K-Means algorithm
The K-Means algorithm is a clustering method that is popular because of its speed and
scalability. K-Means is an iterative process of moving the centers of the clusters, or the
centroids, to the mean position of their constituent points, and re-assigning instances
to their closest clusters. The titular K is a hyperparameter that specifies the number of
clusters that should be created; K-Means automatically assigns observations to clusters
but cannot determine the appropriate number of clusters. K must be a positive integer
that is less than the number of instances in the training set. Sometimes, the number of
clusters is specified by the clustering problem's context. For example, a company that
manufactures shoes might know that it is able to support manufacturing three new
models. To understand what groups of customers to target with each model, it surveys
customers and creates three clusters from the results. That is, the value of K was
specified by the problem's context. Other problems may not require a specific number
of clusters, and the optimal number of clusters may be ambiguous. We will discuss a
heuristic to estimate the optimal number of clusters called the elbow method later in
this chapter.
The parameters of K-Means are the positions of the clusters' centroids and the
observations that are assigned to each cluster. Like generalized linear models and
decision trees, the optimal values of K-Means' parameters are found by minimizing
a cost function. The cost function for K-Means is given by the following equation:
2
1 k
K
i k
k i C
J x μ
= 
= −
www.it-ebooks.info
Clustering with K-Means
[ 118 ]
In the preceding equation, k
μ is the centroid for the cluster k . The cost function sums
the distortions of the clusters. Each cluster's distortion is equal to the sum of the
squared distances between its centroid and its constituent instances. The distortion is
small for compact clusters and large for clusters that contain scattered instances. The
parameters that minimize the cost function are learned through an iterative process
of assigning observations to clusters and then moving the clusters. First, the clusters'
centroids are initialized to random positions. In practice, setting the centroids'
positions equal to the positions of randomly selected observations yields the best
results. During each iteration, K-Means assigns observations to the cluster that they
are closest to, and then moves the centroids to their assigned observations' mean
location. Let's work through an example by hand using the training data shown in
the following table:
Instance X0 X1
1 7 5
2 5 7
3 7 7
4 3 3
5 4 6
6 1 4
7 0 0
8 2 2
9 8 7
10 6 8
11 5 5
12 3 7
There are two explanatory variables and each instance has two features. The instances
are plotted in the following figure:
www.it-ebooks.info
Chapter 6
[ 119 ]
Assume that K-Means initializes the centroid for the first cluster to the fifth instance
and the centroid for the second cluster to the eleventh instance. For each instance,
we will calculate its distance to both centroids, and assign it to the cluster with the
closest centroid. The initial assignments are shown in the Cluster column of the
following table:
Instance X0 X1 C1 distance C2 distance Last cluster Cluster Changed?
1 7 5 3.16228 2 None C2 Yes
2 5 7 1.41421 2 None C1 Yes
3 7 7 3.16228 2.82843 None C2 Yes
4 3 3 3.16228 2.82843 None C2 Yes
5 4 6 0 1.41421 None C1 Yes
6 1 4 3.60555 4.12311 None C1 Yes
7 0 0 7.21110 7.07107 None C2 Yes
8 2 2 4.47214 4.24264 None C2 Yes
9 8 7 4.12311 3.60555 None C2 Yes
10 6 8 2.82843 3.16228 None C1 Yes
11 5 5 1.41421 0 None C2 Yes
12 3 7 1.41421 2.82843 None C1 Yes
C1 centroid 4 6
C2 centroid 5 5
www.it-ebooks.info
Clustering with K-Means
[ 120 ]
The plotted centroids and the initial cluster assignments are shown in the following
graph. Instances assigned to the first cluster are marked with Xs, and instances
assigned to the second cluster are marked with dots. The markers for the centroids
are larger than the markers for the instances.
Now we will move both centroids to the means of their constituent instances,
recalculate the distances of the training instances to the centroids, and reassign the
instances to the closest centroids:
Instance X0 X1 C1
distance
C2
distance
Last
Cluster
New
Cluster Changed?
1 7 5 3.492850 2.575394 C2 C2 No
2 5 7 1.341641 2.889107 C1 C1 No
3 7 7 3.255764 3.749830 C2 C1 Yes
4 3 3 3.492850 1.943067 C2 C2 No
5 4 6 0.447214 1.943067 C1 C1 No
6 1 4 3.687818 3.574285 C1 C2 Yes
7 0 0 7.443118 6.169378 C2 C2 No
www.it-ebooks.info
Chapter 6
[ 121 ]
Instance X0 X1 C1
distance
C2
distance
Last
Cluster
New
Cluster Changed?
8 2 2 4.753946 3.347250 C2 C2 No
9 8 7 4.242641 4.463000 C2 C1 Yes
10 6 8 2.720294 4.113194 C1 C1 No
11 5 5 1.843909 0.958315 C2 C2 No
12 3 7 1 3.260775 C1 C1 No
C1
centroid 3.8 6.4
C2
centroid 4.571429 4.142857
The new clusters are plotted in the following graph. Note that the centroids are
diverging and several instances have changed their assignments:
www.it-ebooks.info
Clustering with K-Means
[ 122 ]
Now, we will move the centroids to the means of their constituents' locations again
and reassign the instances to their nearest centroids. The centroids continue to
diverge, as shown in the following figure:
None of the instances' centroid assignments will change in the next iteration;
K-Means will continue iterating until some stopping criteria is satisfied. Usually,
this criterion is either a threshold for the difference between the values of the cost
function for subsequent iterations, or a threshold for the change in the positions
of the centroids between subsequent iterations. If these stopping criteria are small
enough, K-Means will converge on an optimum. This optimum will not necessarily
be the global optimum.
www.it-ebooks.info
Chapter 6
[ 123 ]
Local optima
Recall that K-Means initially sets the positions of the clusters' centroids to the
positions of randomly selected observations. Sometimes, the random initialization
is unlucky and the centroids are set to positions that cause K-Means to converge to a
local optimum. For example, assume that K-Means randomly initializes two cluster
centroids to the following positions:
www.it-ebooks.info
Clustering with K-Means
[ 124 ]
K-Means will eventually converge on a local optimum like that shown in the
following figure. These clusters may be informative, but it is more likely that the
top and bottom groups of observations are more informative clusters. To avoid
local optima, K-Means is often repeated dozens or even hundreds of times. In
each iteration, it is randomly initialized to different starting cluster positions. The
initialization that minimizes the cost function best is selected.
The elbow method
If K is not specified by the problem's context, the optimal number of clusters can
be estimated using a technique called the elbow method. The elbow method plots
the value of the cost function produced by different values of K. As K increases,
the average distortion will decrease; each cluster will have fewer constituent
instances, and the instances will be closer to their respective centroids. However,
the improvements to the average distortion will decline as K increases. The value
of K at which the improvement to the distortion declines the most is called the
elbow. Let's use the elbow method to choose the number of clusters for a dataset.
The following scatter plot visualizes a dataset with two obvious clusters:
www.it-ebooks.info
Chapter 6
[ 125 ]
We will calculate and plot the mean distortion of the clusters for each value of K
from 1 to 10 with the following code:

The average distortion improves rapidly as we increase K from 1 to 2. There is little
improvement for values of K greater than 2. Now let's use the elbow method on the
following dataset with three clusters:

%=============================================================%
Chapter 6
[ 127 ]
The following figure shows the elbow plot for the dataset. From this, we can see that
the rate of improvement to the average distortion declines the most when adding a
fourth cluster, that is, the elbow method confirms that K should be set to three for
this dataset.
%=============================================================%
Clustering with K-Means
[ 128 ]
Evaluating clusters
We defined machine learning as the design and study of systems that learn from
experience to improve their performance of a task as measured by a given metric.
K-Means is an unsupervised learning algorithm; there are no labels or ground truth
to compare with the clusters. However, we can still evaluate the performance of
the algorithm using intrinsic measures. We have already discussed measuring the
distortions of the clusters. In this section, we will discuss another performance measure
for clustering called the silhouette coefficient. The silhouette coefficient is a measure of
the compactness and separation of the clusters. It increases as the quality of the clusters
increase; it is large for compact clusters that are far from each other and small for large,
overlapping clusters. The silhouette coefficient is calculated per instance; for a set of
instances, it is calculated as the mean of the individual samples' scores. The silhouette
coefficient for an instance is calculated with the following equation:
max ( , )
s ba
a b
=
a is the mean distance between the instances in the cluster. b is the mean distance
between the instance and the instances in the next closest cluster. The following
example runs K-Means four times to create two, three, four, and eight clusters
from a toy dataset and calculates the silhouette coefficient for each run:
>>> import numpy as np
>>> from sklearn.cluster import KMeans
>>> from sklearn import metrics
>>> import matplotlib.pyplot as plt
>>> plt.subplot(3, 2, 1)
>>> x1 = np.array([1, 2, 3, 1, 5, 6, 5, 5, 6, 7, 8, 9, 7, 9])
>>> x2 = np.array([1, 3, 2, 2, 8, 6, 7, 6, 7, 1, 2, 1, 1, 3])
>>> X = np.array(zip(x1, x2)).reshape(len(x1), 2)
>>> plt.xlim([0, 10])
>>> plt.ylim([0, 10])
>>> plt.title('Instances')
>>> plt.scatter(x1, x2)
>>> colors = ['b', 'g', 'r', 'c', 'm', 'y', 'k', 'b']
>>> markers = ['o', 's', 'D', 'v', '^', 'p', '*', '+']
>>> tests = [2, 3, 4, 5, 8]
>>> subplot_counter = 1
>>> for t in tests:
>>> subplot_counter += 1
>>> plt.subplot(3, 2, subplot_counter)
>>> kmeans_model = KMeans(n_clusters=t).fit(X)

%=============================================================%
Chapter 6
[ 129 ]
>>> for i, l in enumerate(kmeans_model.labels_):
>>> plt.plot(x1[i], x2[i], color=colors[l], marker=markers[l],
ls='None')
>>> plt.xlim([0, 10])
>>> plt.ylim([0, 10])
>>> plt.title('K = %s, silhouette coefficient = %.03f' % (
>>> t, metrics.silhouette_score(X, kmeans_model.labels_,
metric='euclidean')))
>>> plt.show()
This script produces the following figure:
www.it-ebooks.info
Clustering with K-Means
[ 130 ]
The dataset contains three obvious clusters. Accordingly, the silhouette coefficient
is greatest when K is equal to three. Setting K equal to eight produces clusters of
instances that are as close to each other as they are to the instances in some of the
other clusters, and the silhouette coefficient of these clusters is smallest.
Image quantization
In the previous sections, we used clustering to explore the structure of a dataset.
Now let's apply it to a different problem. Image quantization is a lossy compression
method that replaces a range of similar colors in an image with a single color.
Quantization reduces the size of the image file since fewer bits are required to
represent the colors. In the following example, we will use clustering to discover a
compressed palette for an image that contains its most important colors. We will then
rebuild the image using the compressed palette. This example requires the mahotas
image processing library, which can be installed using pip install mahotas:
>>> import numpy as np
>>> import matplotlib.pyplot as plt
>>> from sklearn.cluster import KMeans
>>> from sklearn.utils import shuffle
>>> import mahotas as mh
First we read and flatten the image:
>>> original_img = np.array(mh.imread('img/tree.jpg'), dtype=np.
float64) / 255
>>> original_dimensions = tuple(original_img.shape)
>>> width, height, depth = tuple(original_img.shape)
>>> image_flattened = np.reshape(original_img, (width * height, depth))
We then use K-Means to create 64 clusters from a sample of 1,000 randomly selected
colors. Each of the clusters will be a color in the compressed palette. The code is
as follows:
>>> image_array_sample = shuffle(image_flattened, random_state=0)
[:1000]
>>> estimator = KMeans(n_clusters=64, random_state=0)
>>> estimator.fit(image_array_sample)
Next, we predict the cluster assignment for each of the pixels in the original image:
>>> cluster_assignments = estimator.predict(image_flattened)
www.it-ebooks.info
Chapter 6
[ 131 ]
Finally, we create the compressed image from the compressed palette and
cluster assignments:
>>> compressed_palette = estimator.cluster_centers_
>>> compressed_img = np.zeros((width, height, compressed_palette.
shape[1]))
>>> label_idx = 0
>>> for i in range(width):
>>> for j in range(height):
>>> compressed_img[i][j] = compressed_palette[cluster_
assignments[label_idx]]
>>> label_idx += 1
>>> plt.subplot(122)
>>> plt.title('Original Image')
>>> plt.imshow(original_img)
>>> plt.axis('off')
>>> plt.subplot(121)
>>> plt.title('Compressed Image')
>>> plt.imshow(compressed_img)
>>> plt.axis('off')
>>> plt.show()
The original and compressed versions of the image are show in the following figure:
www.it-ebooks.info
Clustering with K-Means
[ 132 ]
Clustering to learn features
In this example, we will combine clustering with classification in a semi-supervised
learning problem. You will learn features by clustering unlabeled data and use the
learned features to build a supervised classifier.
Suppose you own a cat and a dog. Suppose that you have purchased a smartphone,
ostensibly to use to communicate with humans, but in practice just to use to
photograph your cat and dog. Your photographs are awesome and you are certain
that your friends and co-workers would love to review all of them in detail. You'd
like to be courteous and respect that some people will only want to see your cat
photos, while others will only want to see your dog photos, but separating the
photos is laborious. Let's build a semi-supervised learning system that can classify
images of cats and dogs.
Recall from Chapter 3, Feature Extraction and Preprocessing, that a naïve approach
to classifying images is to use the intensities, or brightnesses, of all of the pixels as
explanatory variables. This approach produces high-dimensional feature vectors for
even small images. Unlike the high-dimensional feature vectors we used to represent
documents, these vectors are not sparse. Furthermore, it is obvious that this approach
is sensitive to the image's illumination, scale, and orientation. In Chapter 3, Feature
Extraction and Preprocessing, we also discussed SIFT and SURF descriptors, which
describe interesting regions of an image in ways that are invariant to scale, rotation,
and illumination. In this example, we will cluster the descriptors extracted from all of
the images to learn features. We will then represent an image with a vector with one
element for each cluster. Each element will encode the number of descriptors extracted
from the image that were assigned to the cluster. This approach is sometimes called
the bag-of-features representation, as the collection of clusters is analogous to the
bag-of-words representation's vocabulary. We will use 1,000 images of cats and 1,000
images of dogs from the training set for Kaggle's Dogs vs. Cats competition. The dataset
can be downloaded from https://www.kaggle.com/c/dogs-vs-cats/data. We
will label cats as the positive class and dogs as the negative class. Note that the images
have different dimensions; since our feature vectors do not represent pixels, we do not
need to resize the images to have the same dimensions. We will train using the first 60
percent of the images, and test on the remaining 40 percent:
>>> import numpy as np
>>> import mahotas as mh
>>> from mahotas.features import surf
>>> from sklearn.linear_model import LogisticRegression
>>> from sklearn.metrics import *
>>> from sklearn.cluster import MiniBatchKMeans
>>> import glob
www.it-ebooks.info
Chapter 6
[ 133 ]
First, we load the images, convert them to grayscale, and extract the SURF
descriptors. SURF descriptors can be extracted more quickly than many similar
features, but extracting descriptors from 2,000 images is still computationally
expensive. Unlike the previous examples, this script requires several minutes
to execute on most computers:
>>> all_instance_filenames = []
>>> all_instance_targets = []
>>> for f in glob.glob('cats-and-dogs-img/*.jpg'):
>>> target = 1 if 'cat' in f else 0
>>> all_instance_filenames.append(f)
>>> all_instance_targets.append(target)
>>> surf_features = []
>>> counter = 0
>>> for f in all_instance_filenames:
>>> print 'Reading image:', f
>>> image = mh.imread(f, as_grey=True)
>>> surf_features.append(surf.surf(image)[:, 5:])
>>> train_len = int(len(all_instance_filenames) * .60)
>>> X_train_surf_features = np.concatenate(surf_features[:train_len])
>>> X_test_surf_feautres = np.concatenate(surf_features[train_len:])
>>> y_train = all_instance_targets[:train_len]
>>> y_test = all_instance_targets[train_len:]
We then group the extracted descriptors into 300 clusters in the following code
sample. We use MiniBatchKMeans, a variation of K-Means that uses a random
sample of the instances in each iteration. As it computes the distances to the
centroids for only a sample of the instances in each iteration, MiniBatchKMeans
converges more quickly but its clusters' distortions may be greater. In practice,
the results are similar, and this compromise is acceptable.:
>>> n_clusters = 300
>>> print 'Clustering', len(X_train_surf_features), 'features'
>>> estimator = MiniBatchKMeans(n_clusters=n_clusters)
>>> estimator.fit_transform(X_train_surf_features)
www.it-ebooks.info
Clustering with K-Means
[ 134 ]
Next, we construct feature vectors for the training and testing data. We find the
cluster associated with each of the extracted SURF descriptors, and count them using
NumPy's binCount() function. The following code produces a 300-dimensional
feature vector for each instance:
>>> X_train = []
>>> for instance in surf_features[:train_len]:
>>> clusters = estimator.predict(instance)
>>> features = np.bincount(clusters)
>>> if len(features) < n_clusters:
>>> features = np.append(features, np.zeros((1, n_clusterslen(
features))))
>>> X_train.append(features)
>>> X_test = []
>>> for instance in surf_features[train_len:]:
>>> clusters = estimator.predict(instance)
>>> features = np.bincount(clusters)
>>> if len(features) < n_clusters:
>>> features = np.append(features, np.zeros((1, n_clusterslen(
features))))
>>> X_test.append(features)
Finally, we train a logistic regression classifier on the feature vectors and targets,
and assess its precision, recall, and accuracy:
>>> clf = LogisticRegression(C=0.001, penalty='l2')
>>> clf.fit_transform(X_train, y_train)
>>> predictions = clf.predict(X_test)
>>> print classification_report(y_test, predictions)
>>> print 'Precision: ', precision_score(y_test, predictions)
>>> print 'Recall: ', recall_score(y_test, predictions)
>>> print 'Accuracy: ', accuracy_score(y_test, predictions)
Reading image: dog.9344.jpg
...
Reading image: dog.8892.jpg
www.it-ebooks.info
Chapter 6
[ 135 ]
Clustering 756914 features
precision recall f1-score support
0 0.71 0.76 0.73 392
1 0.75 0.70 0.72 408
avg / total 0.73 0.73 0.73 800
Precision: 0.751322751323
Recall: 0.696078431373
Accuracy: 0.7275
This semi-supervised system has better precision and recall than a logistic regression
classifier that uses only the pixel intensities as features. Furthermore, our feature
representations have only 300 dimensions; even small 100 x 100 pixel images would
have 10,000 dimensions.
Summary
In this chapter, we discussed our first unsupervised learning task: clustering.
Clustering is used to discover structure in unlabeled data. You learned about the
K-Means clustering algorithm, which iteratively assigns instances to clusters and
refines the positions of the cluster centroids. While K-Means learns from experience
without supervision, its performance is still measurable; you learned to use distortion
and the silhouette coefficient to evaluate clusters. We applied K-Means to two
different problems. First, we used K-Means for image quantization, a compression
technique that represents a range of colors with a single color. We also used K-Means
to learn features in a semi-supervised image classification problem.
In the next chapter, we will discuss another unsupervised learning task called
dimensionality reduction. Like the semi-supervised feature representations we
created to classify images of cats and dogs, dimensionality reduction can be used
to reduce the dimensions of a set of explanatory variables while retaining as much
information as possible.
www.it