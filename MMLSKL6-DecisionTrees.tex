Nonlinear Classification
and Regression with
Decision Trees
In the previous chapters we discussed generalized linear models, which relate a
linear combination of explanatory variables to one or more response variables using
a link function. You learned to use multiple linear regression to solve regression
problems, and we used logistic regression for classification tasks. In this chapter we
will discuss a simple, nonlinear model for classification and regression tasks: the
decision tree. We'll use decision trees to build an ad blocker that can learn to classify
images on a web page as banner advertisements or page content. Finally, we will
introduce ensemble learning methods, which combine a set of models to produce an
estimator with better predictive performance than any of its component estimators.
Decision trees
Decision trees are tree-like graphs that model a decision. They are analogous to the
parlor game Twenty Questions. In Twenty Questions, one player, called the answerer,
chooses an object but does not reveal the object to the other players, who are called
questioners. The object should be a common noun, such as "guitar" or "sandwich", but
not "1969 Gibson Les Paul Custom" or "North Carolina". The questioners must guess
the object by asking as many as twenty questions that can be answered with yes, no, or
maybe. An intuitive strategy for questioners is to ask questions of increasing specificity;
asking "is it a musical instrument?" as the first question will not efficiently reduce the
number of possibilities. The branches of a decision tree specify the shortest sequences
of explanatory variables that can be examined in order to estimate the value of a
response variable. To continue the analogy, in Twenty Questions the questioner and
the answerers all have knowledge of the training data, but only the answerer knows
the values of the features for the test instance.
www.it-ebooks.info
Nonlinear Classification and Regression with Decision Trees
[ 98 ]
Decision trees are commonly learned by recursively splitting the set of training
instances into subsets based on the instances' values for the explanatory variables.
The following diagram depicts a decision tree that we will look at in more detail
later in the chapter.
Represented by boxes, the interior nodes of the decision tree test explanatory
variables. These nodes are connected by edges that specify the possible outcomes of
the tests. The training instances are divided into subsets based on the outcomes of
the tests. For example, a node might test whether or not the value of an explanatory
variable exceeds a threshold. The instances that pass the test will follow an edge
to the node's right child, and the instances that fail the test will follow an edge to
the node's left child. The children nodes similarly test their subsets of the training
instances until a stopping criterion is satisfied. In classification tasks, the leaf nodes
of the decision tree represent classes. In regression tasks, the values of the response
variable for the instances contained in a leaf node may be averaged to produce the
estimate for the response variable. After the decision tree has been constructed,
making a prediction for a test instance requires only following the edges until a
leaf node is reached.
www.it-ebooks.info
Chapter 5
[ 99 ]
Training decision trees
Let's create a decision tree using an algorithm called Iterative Dichotomiser 3 (ID3).
Invented by Ross Quinlan, ID3 was one of the first algorithms used to train decision
trees. Assume that you have to classify animals as cats or dogs. Unfortunately, you
cannot observe the animals directly and must use only a few attributes of the animals
to make your decision. For each animal, you are told whether or not it likes to play
fetch, whether or not it is frequently grumpy, and its favorite of three types of food.
To classify new animals, the decision tree will examine an explanatory variable at
each node. The edge it follows to the next node will depend on the outcome of the test.
For example, the first node might ask whether or not the animal likes to play fetch. If
the animal does, we will follow the edge to the left child node; if not, we will follow
the edge to the right child node. Eventually an edge will connect to a leaf node that
indicates whether the animal is a cat or a dog.
The following fourteen instances comprise our training data:
Training instance Plays fetch Is grumpy Favorite food Species
1 Yes No Bacon Dog
2 No Yes Dog Food Dog
3 No Yes Cat food Cat
4 No Yes Bacon Cat
5 No No Cat food Cat
6 No Yes Bacon Cat
7 No Yes Cat Food Cat
8 No No Dog Food Dog
9 No Yes Cat food Cat
10 Yes No Dog Food Dog
11 Yes No Bacon Dog
12 No No Cat food Cat
13 Yes Yes Cat food Cat
14 Yes Yes Bacon Dog
www.it-ebooks.info
Nonlinear Classification and Regression with Decision Trees
[ 100 ]
From this data we can see that cats are generally grumpier than the dogs. Most dogs
play fetch and most cats refuse. Dogs prefer dog food and bacon, whereas cats only
like cat food and bacon. The is grumpy and plays fetch explanatory variables
can be easily converted to binary-valued features. The favorite food explanatory
variable is a categorical variable that has three possible values; we will one-hot encode
it. Recall from Chapter 3, Feature Extraction and Preprocessing, that one-hot encoding
represents a categorical variable with as many binary-valued features as there are
values for variable. Representing the categorical variable with a single integer-valued
feature will encode an artificial order to its values. Since favorite food has three
possible states, we will represent it with three binary-valued features. From this table,
we can manually construct classification rules. For example, an animal that is grumpy
and likes cat food must be a cat, while an animal that plays fetch and likes bacon must
be a dog. Constructing these classification rules by hand for even a small data set is
cumbersome. Instead, we will learn these rules by creating a decision tree.
Selecting the questions
Like Twenty Questions, the decision tree will estimate the value of the response
variable by testing the values of a sequence of explanatory variables. Which
explanatory variable should be tested first? Intuitively, a test that produces subsets that
contain all cats or all dogs is better than a test that produces subsets that still contain
both cats and dogs. If the members of a subset are of different classes, we are still
uncertain about how to classify the instance. We should also avoid creating tests that
separate only a single cat or dog from the others; such tests are analogous to asking
specific questions in the first few rounds of Twenty Questions. More formally, these
tests can infrequently classify an instance and might not reduce our uncertainty. The
tests that reduce our uncertainty about the classification the most are the best. We can
quantify the amount of uncertainty using a measure called entropy.
Measured in bits, entropy quantifies the amount of uncertainty in a variable. Entropy
is given by the following equation, where n is the number of outcomes and ( ) i P x is
the probability of the outcome i. Common values for b are 2, e, and 10. Because the
log of a number less than one will be negative, the entire sum is negated to return a
positive value.
( ) ( ) ( )
1
log
n
i b i
i
H X P x P x
=
= −
www.it-ebooks.info
Chapter 5
[ 101 ]
For example, a single toss of a fair coin has only two outcomes: heads and tails. The
probability that the coin will land on heads is 0.5, and the probability that it will land
on tails is 0.5. The entropy of the coin toss is equal to the following:
H ( X ) = −(0.5log2 0.5 + 0.5log2 0.5) =1.0
That is, only one bit is required to represent the two equally probable outcomes,
heads and tails. Two tosses of a fair coin can result in four possible outcomes: heads
and heads, heads and tails, tails and heads, and tails and tails. The probability of
each outcome is 0.5 x 0.5 = 0.25. The entropy of two tosses is equal to the following:
( ) ( ) 2 2 2 2 H X = − 0.25log 0.25 + 0.25log 0.25 + 0.25log 0.25 + 0.25log 0.25 = 2.0
If the coin has the same face on both sides, the variable representing its outcome has
0 bits of entropy; that is, we are always certain of the outcome and the variable will
never represent new information. Entropy can also be represented as a fraction of
a bit. For example, an unfair coin has two different faces, but is weighted such that
the faces are not equally likely to land in a toss. Assume that the probability that an
unfair coin will land on heads is 0.8, and the probability that it will land on tails is
0.2. The entropy of a single toss of this coin is equal to the following:
( ) ( ) 2 2 H X = − 0.9log 0.9 + 0.2log 0.2 = 0.7219280948873623
The outcome of a single toss of an unfair coin can have a fraction of one bit of
entropy. There are two possible outcomes of the toss, but we are not totally uncertain
since one outcome is more frequent.
Let's calculate the entropy of classifying an unknown animal. If an equal number of
dogs and cats comprise our animal classification training data and we do not know
anything else about the animal, the entropy of the decision is equal to one. All we
know is that the animal could be either a cat or a dog; like the fair coin toss, both
outcomes are equally likely. Our training data, however, contains six dogs and eight
cats. If we do not know anything else about the unknown animal, the entropy of the
decision is given by the following:
( ) 2 2 2
6 log 6 log 8 log 8 0.985228136.342516
14 14 14 14
H X = − +  =
 
www.it-ebooks.info
Nonlinear Classification and Regression with Decision Trees
[ 102 ]
Since cats are more common, we are less uncertain about the outcome. Now let's find
the explanatory variable that will be most helpful in classifying the animal; that is,
let's find the explanatory variable that reduces the entropy the most. We can test the
plays fetch explanatory variable and divide the training instances into animals
that play fetch and animals that don't. This produces the two following subsets:
Decision trees are often visualized as diagrams that are similar to flowcharts. The
top box of the previous diagram is the root node; it contains all of our training
instances and specifies the explanatory variable that will be tested. At the root node
we have not eliminated any instances from the training set and the entropy is equal
to approximately 0.985. The root node tests the plays fetch explanatory variable.
Recall that we converted this Boolean explanatory variable to a binary-valued
feature. Training instances for which plays fetch is equal to zero follow the edge
to the root's left child, and training instances for animals that do play fetch follow
the edge to the root's right child node. The left child node contains a subset of the
training data with seven cats and two dogs that do not like to play fetch. The entropy
at this node is given by the following:
( ) 2 2 2
2 log 2 log 7 log 7 0.7642045065086203
9 9 9 9
H X = − +  =
 
The right child contains a subset with one cat and four dogs that do like to play fetch.
The entropy at this node is given by the following:
( ) 2 2
1 log 1 4 log 4 0.7219280948873623
5 5 5 5
H X = − +  =
 
www.it-ebooks.info
Chapter 5
[ 103 ]
Instead of testing the plays fetch explanatory variable, we could test the is
grumpy explanatory variable. This test produces the following tree. As with the
previous tree, instances that fail the test follow the left edge, and instances that
pass the test follow the right edge.
We could also divide the instances into animals that prefer cat food and animals that
don't to produce the following tree:
Information gain
Testing for the animals that prefer cat food resulted in one subset with six cats, zero
dogs, and 0 bits of entropy and another subset with two cats, six dogs, and 0.811
bits of entropy. How can we measure which of these tests reduced our uncertainty
about the classification the most? Averaging the entropies of the subsets may seem to
be an appropriate measure of the reduction in entropy. In this example, the subsets
produced by the cat food test have the lowest average entropy. Intuitively, this test
seems to be effective, as we can use it to classify almost half of the training instances.
However, selecting the test that produces the subsets with the lowest average
entropy can produce a suboptimal tree. For example, imagine a test that produced
one subset with two dogs and no cats and another subset with four dogs and eight
cats. The entropy of the first subset is equal to the following (note that the second
term is omitted because 2 log 0 is undefined):
( ) 2
2 log 2 0.0
2 2
H X = −  =
 
www.it-ebooks.info
Nonlinear Classification and Regression with Decision Trees
[ 104 ]
The entropy of the second subset is equal to the following:
( ) 2 2
4 log 4 8 log 8 0.9182958340544896
12 12 12 12
H X = − +  =
 
The average of these subsets' entropies is only 0.459, but the subset containing
most of the instances has almost one bit of entropy. This is analogous to asking
specific questions early in Twenty Questions; we could get lucky and win within
the first few attempts, but it is more likely that we will squander our questions
without eliminating many possibilities. Instead, we will measure the reduction
in entropy using a metric called information gain. Calculated with the following
equation, information gain is the difference between the entropy of the parent
node, H (T ), and the weighted average of the children nodes' entropies. T is
the set of instances, and a is the explanatory variable under test. ( ) a x vals a
is the value of attribute a for instance x. { | } a xT x = v is the number
of instances for which attribute a is equal to the value v. ({ | }) a H xT x = v
is the entropy of the subset of instances for which the value of the explanatory
variable a is v.
( ) ( ) { }
( )
({ }) |
, | a
a
v vals a
x T x v
IG T a H T H x T x v
 T
 =
= −   =
The following table contains the information gains for all of the tests. In this case, the
cat food test is still the best, as it increases the information gain the most.
Test Parent's
entropy
Child's
entropy
Child's
entropy Weighted average IG
plays fetch? 0.9852 0.7642 0.7219 0.7490 * 9/14 + 0.7219 * 5/14
= 0.7491 0.2361
is grumpy? 0.9852 0.9183 0.8113 0.9183 * 6/14 + 0.8113 * 8/14
= 0.85710.8572 0.1280
favorite food =
cat food 0.9852 0.8113 0 0.8113 * 8 /14 + 0.0 * 6/14 =
0.4636 0.5216
favorite food =
dog food 0.9852 0.8454 0 0.8454 * 11/14 + 0.0 * 3/14 =
0.6642 0.3210
favorite food =
bacon 0.9852 0.9183 0.971 0.9183 * 9/14 + 0.9710 * 5/14
= 0.9371 0.0481
www.it-ebooks.info
Chapter 5
[ 105 ]
Now let's add another node to the tree. One of the child nodes produced by the test
is a leaf node that contains only cats. The other node still contains two cats and six
dogs. We will add a test to this node. Which of the remaining explanatory variables
reduces our uncertainty the most? The following table contains the information gains
for all of the possible tests:
Test Parent's
entropy
Child's
entropy
Child's
entropy Weighted average IG
plays fetch? 0.8113 1 0 1.0 * 4/8 + 0 * 4/8 = 0.5 0.3113
is grumpy? 0.8113 0 1 0.0 * 4/8 + 1 * 4/8 = 0.5 0.3113
favorite
food=dog food 0.8113 0.9710 0 0.9710 * 5/8 + 0.0 * 3/8
= 0.6069 0.2044
favorite
food=bacon 0.8113 0 0.9710 0.0 * 3/8 + 0.9710 * 5/8
= 0.6069 0.2044
All of the tests produce subsets with 0 bits of entropy, but the is grumpy and plays
fetch tests produce the greatest information gain. ID3 breaks ties by selecting one of
the best tests arbitrarily. We will select the is grumpy test, which splits its parent's
eight instances into a leaf node containing four dogs and a node containing two cats
and two dogs. The following is a diagram of the current tree:
www.it-ebooks.info
Nonlinear Classification and Regression with Decision Trees
[ 106 ]
We will now select another explanatory variable to test the child node's four
instances. The remaining tests, favorite food=bacon, favorite food=dog food,
and plays fetch, all produce a leaf node containing one dog or cat and a node
containing the remaining animals. The remaining tests produce equal information
gains, as shown in the following table:
Test Parent's
Entropy
Child
Entropy
Child
Entropy
Weighted
Average
Information
Gain
plays fetch? 1 0.9183 0 0.688725 0.311275
favorite
food=dog
food
1 0.9183 0 0.688725 0.311275
favorite
food=bacon 1 0 0.9183 0.688725 0.311275
We will arbitrarily select the plays fetch test to produce a leaf node containing one
dog and a node containing two cats and a dog. Two explanatory variables remain;
we can test for animals that like bacon, or we can test for animals that like dog food.
Both of the tests will produce the same subsets and create a leaf node containing one
dog and a leaf node containing two cats. We will arbitrarily choose to test for animals
that like dog food. The following is a diagram of the completed decision tree:
www.it-ebooks.info
Chapter 5
[ 107 ]
Let's classify some animals from the following test data:
Testing instance Plays fetch Is grumpy Favorite food Species
1 Yes No Bacon Dog
2 Yes Yes Dog Food Dog
3 No Yes Dog Food Cat
4 No Yes Bacon Cat
5 No No Cat food Cat
www.it-ebooks.info
Nonlinear Classification and Regression with Decision Trees
[ 108 ]
Let's classify the first animal, which likes to plays fetch, is infrequently grumpy, and
loves bacon. We will follow the edge to the root node's left child since the animal's
favorite food is not cat food. The animal is not grumpy, so we will follow the edge to
the second-level node's left child. This is a leaf node containing only dogs; we have
correctly classified this instance. To classify the third test instance as a cat, we follow
the edge to the root node's left child, follow the edge to the second-level node's right
child, follow the edge to the third-level node's left child, and finally follow the edge
to the fourth-level node's right child.
Congratulations! You've constructed a decision tree using the ID3 algorithm. Other
algorithms can be used to train decision trees. C4.5 is a modified version of ID3
that can be used with continuous explanatory variables and can accommodate
missing values for features. C4.5 also can prune trees. Pruning reduces the size of
a tree by replacing branches that classify few instances with leaf nodes. Used by
scikit-learn's implementation of decision trees, CART is another learning algorithm
that supports pruning.
Gini impurity
In the previous section, we built a decision tree by creating nodes that produced
the greatest information gain. Another common heuristic for learning decision trees
is Gini impurity, which measures the proportions of classes in a set. Gini impurity
is given by the following equation, where j is the number of classes, t is the subset
of instances for the node, and P(i | t ) is the probability of selecting an element of
class i from the node's subset:
( ) ( )2
1
1 |
j
i
Gini t P i t
=
= −
Intuitively, Gini impurity is zero when all of the elements of the set are the same
class, as the probability of selecting an element of that class is equal to one. Like
entropy, Gini impurity is greatest when each class has an equal probability of being
selected. The maximum value of Gini impurity depends on the number of possible
classes, and it is given by the following equation:
max
Gini 1 1
n
= −
www.it-ebooks.info
Chapter 5
[ 109 ]
Our problem has two classes, so the maximum value of the Gini impurity measure
will be equal to one half. scikit-learn supports learning decision trees using both
information gain and Gini impurity. There are no firm rules to help you decide when
to use one criterion or the other; in practice, they often produce similar results. As
with many decisions in machine learning, it is best to compare the performances of
models trained using both options.
Decision trees with scikit-learn
Let's use decision trees to create software that can block banner ads on web pages.
This program will predict whether each of the images on a web page is an
advertisement or article content. Images that are classified as being advertisements
could then be hidden using Cascading Style Sheets. We will train a decision tree
classifier using the Internet Advertisements Data Set from http://archive.ics.uci.
edu/ml/datasets/Internet+Advertisements, which contains data for 3,279 images.
The proportions of the classes are skewed; 459 of the images are advertisements and
2,820 are content. Decision tree learning algorithms can produce biased trees from data
with unbalanced class proportions; we will evaluate a model on the unaltered data set
before deciding if it is worth balancing the training data by over- or under-sampling
instances. The explanatory variables are the dimensions of the image, words from the
containing page's URL, words from the image's URL, the image's alt text, the image's
anchor text, and a window of words surrounding the image tag. The response variable
is the image's class. The explanatory variables have already been transformed into
feature representations. The first three features are real numbers that encode the width,
height, and aspect ratio of the images. The remaining features encode binary term
frequencies for the text variables. In the following sample, we will grid search for the
hyperparameter values that produce the decision tree with the greatest accuracy,
and then evaluate the tree's performance on a test set:
import pandas as pd
from sklearn.tree import DecisionTreeClassifier
from sklearn.cross_validation import train_test_split
from sklearn.metrics import classification_report
from sklearn.pipeline import Pipeline
from sklearn.grid_search import GridSearchCV
www.it-ebooks.info
Nonlinear Classification and Regression with Decision Trees
[ 110 ]
First we read the .csv file using pandas. The .csv does not have a header row,
so we split the last column containing the response variable's values from the
features using its index:
if __name__ == '__main__':
df = pd.read_csv('data/ad.data', header=None)
explanatory_variable_columns = set(df.columns.values)
response_variable_column = df[len(df.columns.values)-1]
# The last column describes the targets
explanatory_variable_columns.remove(len(df.columns.values)-1)
y = [1 if e == 'ad.' else 0 for e in response_variable_column]
X = df[list(explanatory_variable_columns)]
We encoded the advertisements as the positive class and the content as the negative
class. More than one quarter of the instances are missing at least one of the values
for the image's dimensions. These missing values are marked by whitespace and a
question mark. We replaced the missing values with negative one, but we could have
imputed the missing values; for instance, we could have replaced the missing height
values with the average height value:
X.replace(to_replace=' *\?', value=-1, regex=True, inplace=True)
We then split the data into training and test sets:
X_train, X_test, y_train, y_test = train_test_split(X, y)
We created a pipeline and an instance of DecisionTreeClassifier. Then, we set
the criterion keyword argument to entropy to build the tree using the information
gain heuristic:
pipeline = Pipeline([
('clf', DecisionTreeClassifier(criterion='entropy'))
])
Next, we specified the hyperparameter space for the grid search:
parameters = {
	'clf__max_depth': (150, 155, 160),
	'clf__min_samples_split': (1, 2, 3),
	'clf__min_samples_leaf': (1, 2, 3)
}
www.it-ebooks.info
Chapter 5
[ 111 ]
We set GridSearchCV() to maximize the model's F1 score:
grid_search = GridSearchCV(pipeline, parameters, n_jobs=-1,
verbose=1, scoring='f1')
grid_search.fit(X_train, y_train)
print 'Best score: %0.3f' % grid_search.best_score_
print 'Best parameters set:'
best_parameters = grid_search.best_estimator_.get_params()
for param_name in sorted(parameters.keys()):
print '\t%s: %r' % (param_name, best_parameters[param_name])
predictions = grid_search.predict(X_test)
print classification_report(y_test, predictions)
Fitting 3 folds for each of 27 candidates, totalling 81 fits
[Parallel(n_jobs=-1)]: Done 1 jobs | elapsed: 1.7s
[Parallel(n_jobs=-1)]: Done 50 jobs | elapsed: 15.0s
[Parallel(n_jobs=-1)]: Done 71 out of 81 | elapsed: 20.7s
remaining: 2.9s
[Parallel(n_jobs=-1)]: Done 81 out of 81 | elapsed: 23.3s finished
Best score: 0.878
Best parameters set:
clf__max_depth: 155
clf__min_samples_leaf: 2
clf__min_samples_split: 1
precision recall f1-score support
0 0.97 0.99 0.98 710
1 0.92 0.81 0.86 110
avg / total 0.96 0.96 0.96 820
The classifier detected more than 80 percent of the ads in the test set, and
approximately 92 percent of the images that it predicted were ads were truly ads.
Overall, the performance is promising; in following sections, we will try to modify
our model to improve its performance.
www.it-ebooks.info
Nonlinear Classification and Regression with Decision Trees
[ 112 ]
Tree ensembles
Ensemble learning methods combine a set of models to produce an estimator that
has better predictive performance than its individual components. A random forest
is a collection of decision trees that have been trained on randomly selected subsets
of the training instances and explanatory variables. Random forests usually make
predictions by returning the mode or mean of the predictions of their constituent
trees; scikit-learn's implementations return the mean of the trees' predictions.
Random forests are less prone to overfitting than decision trees because no single
tree can learn from all of the instances and explanatory variables; no single tree can
memorize all of the noise in the representation.
Let's update our ad blocker's classifier to use a random forest. It is simple to replace
the DecisionTreeClassifier using scikit-learn's API; we simply replace the object
with an instance of RandomForestClassifier. Like the previous example, we will
grid search to find the values of the hyperparameters that produce the random forest
with the best predictive performance.
First, import the RandomForestClassifier class from the ensemble module:
from sklearn.ensemble import RandomForestClassifier
Next, replace the DecisionTreeClassifier in the pipeline with an instance of
RandomForestClassifier and update the hyperparameter space:
pipeline = Pipeline([
('clf', RandomForestClassifier(criterion='entropy'))
])
parameters = {
	'clf__n_estimators': (5, 10, 20, 50),
	'clf__max_depth': (50, 150, 250),
	'clf__min_samples_split': (1, 2, 3),
	'clf__min_samples_leaf': (1, 2, 3)
}
The output is as follows:
Fitting 3 folds for each of 108 candidates, totalling 324 fits
[Parallel(n_jobs=-1)]: Done 1 jobs | elapsed: 1.1s
[Parallel(n_jobs=-1)]: Done 50 jobs | elapsed: 17.4s
[Parallel(n_jobs=-1)]: Done 200 jobs | elapsed: 1.0min
[Parallel(n_jobs=-1)]: Done 324 out of 324 | elapsed: 1.6min finished
Best score: 0.936
www.it-ebooks.info
Chapter 5
[ 113 ]
Best parameters set:
clf__max_depth: 250
clf__min_samples_leaf: 1
clf__min_samples_split: 3
clf__n_estimators: 20
precision recall f1-score support
0 0.97 1.00 0.98 705
1 0.97 0.83 0.90 115
avg / total 0.97 0.97 0.97 820
Replacing the single decision tree with a random forest resulted in a significant
reduction of the error rate; the random forest improves the precision and recall for
ads to 0.97 and 0.83.
The advantages and disadvantages of
decision trees
The compromises associated with using decision trees are different than those of
the other models we discussed. Decision trees are easy to use. Unlike many learning
algorithms, decision trees do not require the data to have zero mean and unit
variance. While decision trees can tolerate missing values for explanatory variables,
scikit-learn's current implementation cannot. Decision trees can even learn to ignore
explanatory variables that are not relevant to the task.
Small decision trees can be easy to interpret and visualize with the export_graphviz
function from scikit-learn's tree module. The branches of a decision tree are
conjunctions of logical predicates, and they are easily visualized as flowcharts.
Decision trees support multioutput tasks, and a single decision tree can be used for
multiclass classification without employing a strategy like one-versus-all.
Like the other models we discussed, decision trees are eager learners. Eager learners
must build an input-independent model from the training data before they can be
used to estimate the values of test instances, but can predict relatively quickly once
the model has been built. In contrast, lazy learners such as the k-nearest neighbors
algorithm defer all generalization until they must make a prediction. Lazy learners
do not spend time training, but often predict slowly compared to eager learners.
www.it-ebooks.info
Nonlinear Classification and Regression with Decision Trees
[ 114 ]
Decision trees are more prone to overfitting than many of the models we discussed,
as their learning algorithms can produce large, complicated decision trees that
perfectly model every training instance but fail to generalize the real relationship.
Several techniques can mitigate over-fitting in decision trees. Pruning is a common
strategy that removes some of the tallest nodes and leaves of a decision tree, but
it is not currently implemented in scikit-learn. However, similar effects can be
achieved by setting a maximum depth for the tree or by creating child nodes only
when the number of training instances they will contain exceeds a threshold. The
DecisionTreeClassifier and DecisionTreeRegressor classes provide keyword
arguments to set these constraints. Creating a random forest can also reduce
over-fitting.
Efficient decision tree learning algorithms like ID3 are greedy. They learn efficiently
by making locally optimal decisions, but are not guaranteed to produce the globally
optimal tree. ID3 constructs a tree by selecting a sequence of explanatory variables
to test. Each explanatory variable is selected because it reduces the uncertainty in the
node more than the other variables. It is possible, however, that locally suboptimal
tests are required in order to find the globally optimal tree.
In our toy examples, the size of the tree did not matter since we retained all of nodes.
In a real application, however, the tree's growth could be limited by pruning or similar
mechanisms. Pruning trees with different shapes can produce trees with different
performances. In practice, locally optimal decisions that are guided by the information
gain or Gini impurity heuristics often result in an acceptable decision trees.
Summary
In this chapter we learned about simple nonlinear models for classification and
regression called decision trees. Like the parlor game Twenty Questions, decision trees
are composed of sequences of questions that examine a test instance. The branches
of a decision tree terminate in leaves that specify the predicted value of the response
variable. We discussed how to train decision trees using the ID3 algorithm, which
recursively splits the training instances into subsets that reduce our uncertainty about
the value of the response variable. We also discussed ensemble learning methods,
which combine the predictions from a set of models to produce an estimator with
better predictive performance. Finally, we used random forests to predict whether or
not an image on a web page is a banner advertisement. In the next chapter, we will
introduce our first unsupervised learning task: clustering.
\end{document}