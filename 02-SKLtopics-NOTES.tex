\documentclass[SKL-MASTER.tex]{subfiles}

	
\textbf{What does Scikit Learn do?}\\
\begin{figure}[h!]
	\centering
	\includegraphics[width=1.2\linewidth]{images/SKLsite}
	
\end{figure}
\newpage

 \textbf{1. Classification}

*    \textbf{Description:} Identifying to which category an object belongs to.
*    \textbf{Applications:} Spam detection, Image recognition.
*    \textbf{Algorithms:} SVM, nearest neighbors, random forest, 



 \textbf{2. Regression}

*    \textbf{Description:} Predicting a continuous-valued attribute associated with an object.
*    \textbf{Applications:} Drug response, Stock prices.
*    \textbf{Algorithms:} SVR, ridge regression, Lasso, 
	


 \textbf{3. Clustering}


*    \textbf{Description: } Automatic grouping of similar objects into sets.
*    \textbf{Applications:} Customer segmentation, Grouping experiment outcomes
*    \textbf{Algorithms:} k-Means, spectral clustering, mean-shift, ...


\newpage
 \textbf{4. Dimensionality Reduction}


*    \textbf{Description: } Reducing the number of random variables to consider.
*    \textbf{Applications:} Visualization, Increased efficiency
*    \textbf{Algorithms:} PCA, feature selection, non-negative matrix factorization. 


 \textbf{5. Model selection}

*    \textbf{Description: } Comparing, validating and choosing parameters and models.
*    \textbf{Goal:} Improved accuracy via parameter tuning
*    \textbf{Modules:} grid search, cross validation, metrics


 \textbf{6. Preprocessing}

*    \textbf{Description:} Feature extraction and normalization.
*    \textbf{Application:} Transforming input data such as text for use with machine learning algorithms.
*    \textbf{Modules:} preprocessing, feature extraction.

\newpage
\begin{figure}[h!]
\centering
\includegraphics[width=1.2\linewidth]{images/SKLCheatSheet}

\end{figure}
\newpage
\begin{figure}[h!]
	\centering
	\includegraphics[width=1.23\linewidth]{images/SKLCheatSheet2}

\end{figure}
\end{document}