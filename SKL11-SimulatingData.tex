\documentclass[SKL-MASTER.tex]{subfiles}
\begin{document}
	\Large
\section*{Creating sample data for introductory analysis}
We can use scikit-learn to create toy data.
\subsection*{Getting ready}
\begin{itemize}
\item Very similar to getting built-in datasets, fetching new datasets, and creating sample datasets,
the functions that are used follow the naming convention \texttt{make\_$<$the data set$>$}. 
\item Just to
be clear, this data is purely artificial:
\end{itemize}


%===========================================================================================%
%-- Chapter 1
% 11
\begin{framed}
\begin{verbatim}
>>> datasets.make_*?
datasets.make_biclusters
datasets.make_blobs

datasets.make_checkerboard
datasets.make_circles
datasets.make_classification
...
\end{verbatim}
\end{framed}
%\noindent To save typing, import the datasets module as d , and numpy as np:
\begin{framed}
	\begin{verbatim}
>>> import sklearn.datasets as d
>>> import numpy as np
\end{verbatim}
\end{framed}
\newpage
%------------------------ %
\subsection*{Implementation}
%This section will walk you through the creation of several datasets; the following How
%it works... section will confirm the purported characteristics of the datasets. In addition
%to the sample datasets, these will be used throughout the book to create data with the
%necessary characteristics for the algorithms on display.
%
%First, the stalwart—regression:
%------------------------%
\begin{framed}
	\begin{verbatim}
>>> reg_data = d.make_regression()
\end{verbatim}
\end{framed}
%------------------------ %
\begin{itemize}
\item By default, this will generate a tuple with a 100 x 100 matrix : 100 samples by 100 features.
\item By default, only 10 features are responsible for the target data generation. The
second member of the tuple is the target variable.
\item It is also possible to get more involved. For example, to generate a 1000 x 10 matrix with five
features responsible for the target creation, an underlying bias factor of 1.0, and 2 targets,
the following command will be run:
\end{itemize}

%------------------------%
\begin{framed}
	\begin{verbatim}
>>> complex_reg_data =
     d.make_regression(1000, 10, 5, 2, 1.0)
>>>
>>> complex_reg_data[0].shape
(1000, 10)
\end{verbatim}
\end{framed}
%------------------------ %
\newpage
\subsection*{Artificial Data sets for Classification}
\begin{itemize}
\item Classification datasets are also very simple to create. 
\item It's simple to create a base classification
set, but the basic case is rarely experienced in practice—most users don't convert, most
transactions aren't fraudulent, and so on. \item Therefore, it's useful to explore classification on
unbalanced datasets:
\end{itemize}
%------------------------%
\begin{framed}
\begin{verbatim}
>>> classification_set = 
   d.make_classification(weights=[0.1])
>>>
>>> np.bincount(classification_set[1])
array([10, 90])
\end{verbatim}
\end{framed}
%------------------------ %
\newpage
\section*{Clusters}
%Clusters will also be covered. 
\begin{itemize}
\item There are actually several functions to create datasets that can
be modeled by different cluster algorithms. \item For example, blobs are very easy to create and
can be modeled by K-Means:
\end{itemize}
%------------------------%
\begin{framed}
	\begin{verbatim}
>>> blobs = d.make_blobs()
\end{verbatim}
\end{framed}
%------------------------ %
%===========================================================================================%
%Premodel Workflow
%12
\begin{figure}[h!]
\centering
\includegraphics[width=0.6\linewidth]{images/SKL11-data}
\end{figure}
\end{document}
\newpage
Let's walk you through how scikit-learn produces the regression dataset by taking a look at
the source code (with some modifications for clarity). Any undefined variables are assumed
to have the default value of \texttt{make\_regression}.
It's actually surprisingly simple to follow.
First, a random array is generated with the size specified when the function is called:
\begin{framed}
	\begin{verbatim}
>>> X = np.random.randn(n_samples, n_features)
\end{verbatim}
\end{framed}
Given the basic dataset, the target dataset is then generated:
%------------------------%
\begin{framed}
	\begin{verbatim}
>>> ground_truth = np.zeroes((np_samples, n_target))
>>> ground_truth[:n_informative, :] 
     = 100*np.random.rand(n_informative,
         n_targets)
\end{verbatim}
\end{framed}
%------------------------ %
The dot product of X and \texttt{ground\_truth} are taken to get the final target values. Bias, if any,
is added at this time:
%------------------------%
\begin{framed}
\begin{verbatim}
>>> y = np.dot(X, ground_truth) + bias
\end{verbatim}
\end{framed}
%------------------------ %
The dot product is simply a matrix multiplication. So, our final dataset
will have \texttt{n\_samples}, which is the number of rows from the dataset,
and \texttt{n\_target}, which is the number of target variables.
Due to NumPy's broadcasting, bias can be a scalar value, and this value will be added to
every sample.
%===========================================================================================%
%-- Chapter 1
% % - 13
Finally, it's a simple matter of adding any noise and shuffling the dataset. We now have a
dataset perfect to test regression.


\end{document}