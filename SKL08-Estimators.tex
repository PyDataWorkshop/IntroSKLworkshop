\documentclass[SKL-MASTER.tex]{subfiles}



 \textbf{Estimators objects: Fitting data:}\\The core object of scikit-learn is the estimator object. All estimator objects expose a \texttt{fit} method, that takes as input a dataset (2D array):

<pre>
\begin{verbatim}
>>> estimator.fit(data)
\end{verbatim}
\end{framed}

 Suppose \texttt{LogReg} and \texttt{KNN} are (shorthand names for) scikit-learn estimators.
<pre>
\begin{verbatim}
>>> # Supervised Learning Problem
>>> LogReg.fit(SAheartFeat, SAheartTarget)
>>>
>>> # Unsupervised Learning Problem
>>> KNN.fit(IrisFeat)
\end{verbatim}
\end{framed}

\newpage
----------------------%
 \textbf{Estimator parameters:}\\
All the parameters of an estimator can be set when it is instanciated, or by modifying the corresponding attribute:

<pre>
\begin{verbatim}
>>> estimator = Estimator(param1=1, param2=2)
>>> estimator.param1
\end{verbatim}
\end{framed}

----------------------%
% % \subsubsection{Retrieving Estimator parameters:}

 \textbf{Retrieving Estimator parameters:}\\ 

*   When data is fitted with an estimator, parameters are estimated from the data at hand.
*   All the estimated parameters are attributes of the estimator object ending by an underscore:

<pre>
\begin{verbatim}
>>> estimator.estimated_param_ 
\end{verbatim}
\end{framed}
%========================================================================%
\end{document}