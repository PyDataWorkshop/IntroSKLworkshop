\documentclass[SKL-MASTER.tex]{subfiles}

% %- http://gaelvaroquaux.github.io/scikit-learn-tutorial/model_selection.html

\subsubsection{2.2.4. Classification}


% _images/logistic_regression1.png

For classification, as in the labeling iris task, linear regression is not the right approach, as it will give too much weight to data far from the decision frontier. A linear approach is to fit a sigmoid function, or logistic function:

\[y = \textrm{sigmoid}(X\beta - \textrm{offset}) + \epsilon =
\frac{1}{1 + \textrm{exp}(- X\beta + \textrm{offset})} + \epsilon\]

\begin{figure}[h!]
	\centering
	\includegraphics[width=0.7\linewidth]{sklcass/logistic_regression1}
	\caption{}
	\label{fig:logistic_regression1}
\end{figure}

<pre>
\begin{verbatim}
>>> logistic = linear_model.LogisticRegression(C=1e5)
>>> logistic.fit(iris_X_train, iris_y_train)
LogisticRegression(C=100000.0, intercept_scaling=1, dual=False,
          fit_intercept=True, penalty='l2', tol=0.0001)
 
\end{verbatim}
\end{framed}

\begin{figure}
\centering
\includegraphics[width=0.7\linewidth]{sklcass/iris_logistic1}
% % \caption{}
% % \label{fig:iris_logistic1}
\end{figure}

%==================================================================== %
\newpage
\subsection{Multiclass classification}

If you have several classes to predict, an option often used is to fit one-versus-all classifiers, and use a voting heuristic for the final decision.

\subsection{Shrinkage and sparsity with logistic regression}

The C parameter controls the amount of regularization in the LogisticRegression object, the bigger C, the less regularization. penalty=”l2” gives shrinkage (i.e. non-sparse coefficients), while penalty=”l1” gives sparsity.

Excercise

Try classifying the digits dataset with nearest neihbors and a linear model. Leave out the last 10\% and test prediction performance on these observations.
\end{document}